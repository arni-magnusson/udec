\documentclass[aspectratio=169]{beamer}
\usepackage[utf8]{inputenc}
\usepackage[T1]{fontenc}
\usepackage{generic}
\newcommand\cpp{\mbox{C\raisebox{0.5ex}{\tiny\bfseries ++}}}
\newcommand\latex{\mbox{La\h{-0.3ex}TeX}}
\begin{document}

\begin{frame}
  \title{\vspace{-2ex}\darkblue Statistical Computing}
  \author{\vspace{-16ex}\normalsize Arni Magnusson}
  \date{\vspace{-3ex}
    {\darkgreen\it Statistical Modeling in R}\\[0.5ex]
    Universidad de Concepción\\[0.5ex]
    19\h{0.2ex}--\h{0.1ex}23 January 2026}
  \titlepage
\end{frame}

% ______________________________________________________________________________

\begin{frame}{Outline}
  \textbf{\darkblue Statistical Computing and R}\\
  \comment{features, history}\\[4ex]
  \textbf{\darkblue Open Science}\\
  \comment{software, scientific method, repeatability, reviewability}\\[4ex]
  \textbf{\darkblue First Steps in R}\\
  \comment{calculator, objects, plots, help}\\[4ex]
\end{frame}

% ______________________________________________________________________________

\begin{frame}{Statistical computing}
  \begin{tabular}{ll}
    GUI         &~ Excel, JMP, LibreOffice, Minitab, SPSS, Stata     \\[3ex]
    Interpreted &~ BUGS, Julia, Matlab, Python, \textbf{\blue R}, SAS\\[3ex]
    Compiled    &~ C/\cpp, Fortran, Java                             \\[3ex]
    Related     &~ Bash, HTML, \latex, Make, SQL                     \\[3ex]
    Editor      &~ Emacs, Positron, \textbf{\green RStudio}, VS Code \\[2ex]
  \end{tabular}
\end{frame}

% ______________________________________________________________________________

\begin{frame}{Spreadsheets}
  Excel and LibreOffice Calc can be useful for:\\[0.2ex]
  \begin{itemize}
    \item Taking notes in an environment where you can calculate summary
    statistics\\
    \comment{family trip, project management}
    \item Simple statistical analysis, if spreadsheet is the only tool you
    know\\[3ex]
  \end{itemize}
  Compared to statistical software, spreadsheets are:\\[0.2ex]
  \begin{itemize}
    \item[-] Limited, with few features \vspace{-0.2ex}
    \item[-] Unreliable, can produce the wrong result \vspace{-0.2ex}
    \item[-] Error-prone, easy to make mistakes without noticing \vspace{-0.2ex}
    \item[-] Unwieldy, difficult to work with many tables and sheet references
    \vspace{-0.2ex}
    \item[-] Can be difficult to review or repeat analysis for another
    dataset\\[2ex]
  \end{itemize}
  Use only \texttt{+ - / *}, \texttt{sum}, \texttt{average}, and statistical
  software for everything else\\[1ex]
\end{frame}

% ______________________________________________________________________________

\begin{frame}{Using the right tool}
  Imagine writing a 20-page text document in Excel\\[1ex]
  \begin{itemize}
    \item[] $\Rightarrow$ inferior quality, hard to modify, prone to
    errors\\[5ex]
  \end{itemize}
  Likewise, R is not always the right tool in statistical computing:\\[1ex]
  \begin{itemize}
    \item[] \textblue{Databases} for large amounts of data\\[1ex]
    \item[] \textblue{\cpp} for computationally intensive subtasks\\[5ex]
  \end{itemize}
  R has good support for connecting to databases and running compiled \cpp\ code
\end{frame}

% ______________________________________________________________________________

\begin{frame}{R features}
  \begin{itemize}
    \item[] Large collection of tools for statistical analysis, constantly
    updated by\\
    a large user commity, including leading authors in statistical fields\\[5ex]
    \item[] Graphics for exploratory analysis and publications\\[5ex]
    \item[] Language for expressing statistical models, object-oriented and\\
    extensible by users\\[5ex]
    \item[] Used by university statistics departments and research institutes\\
    around the world\\[2ex]
  \end{itemize}
\end{frame}

% ______________________________________________________________________________

\begin{frame}{R early history}
  \textbf{\blue S}\\
  \begin{itemize} \item[]\small
    Programming language, first released in 1976\\
    Created by John Chambers et al., Bell Laboratories\\[3.5ex]
  \end{itemize}
  \textbf{\blue S-Plus}\\
  \begin{itemize} \item[]\small
    Statistical software based on S, first released in 1988\\
    Created by R Douglas Martin, Univ Washington\\[3.5ex]
  \end{itemize}
  \textbf{\blue R}\\
  \begin{itemize} \item[]\small
    Statistical software based on S, first released in 1993\\
    Created by Ross Ihaka and Robert Gentleman, Univ Auckland\\
    Maintained by R Core Development Team since 1997\\
    R became better than S-Plus around 2002\\[1.5ex]
  \end{itemize}
\end{frame}

% ______________________________________________________________________________

\begin{frame}{RStudio and tidyverse}
  \textbf{\green ggplot2}
  \begin{itemize} \item[]\small
    Plotting package, first released in 2007\\
    Created by Hadley Wickham\\[2.5ex]
  \end{itemize}
  \textbf{\green RStudio}
  \begin{itemize} \item[]\small
    Development environment, first released in 2011\\
    Created by J.J. Allaire and Joe Cheng\\[2.5ex]
  \end{itemize}
  \textbf{\green dplyr}
  \begin{itemize} \item[]\small
    Data manipulation package, first released in 2014\\
    Created by Hadley Wickham and Romain François\\[2.5ex]
  \end{itemize}
  \textbf{\green R for Data Science}
  \begin{itemize} \item[]\small
    Influential textbook, first published in 2017\\
    Written by Hadley Wickham and Garrett Grolemund\\[2ex]
  \end{itemize}
\end{frame}

% ______________________________________________________________________________

\begin{frame}{R Project website}
  \begin{center}
    \textblue{\url{https://www.r-project.org}}\\[3ex]
    \textgray{Download R, manuals, etc.}
  \end{center}
\end{frame}

% ______________________________________________________________________________

\begin{frame}
  \begin{center}
    \vspace{-1ex}
    \includegraphics[height=1.05\textheight]{open_science}
  \end{center}
\end{frame}

% ______________________________________________________________________________

\begin{frame}{Open source}
  \begin{itemize}
    \item[] Most R functions are written in the R language, and the\\
    full code is shown if you type the name of the function\\[3ex]
    Low-level functions are written in C, and the full code can\\
    be browsed at \url{https://svn.r-project.org/R/trunk/}\\[3ex]
    This access to the source code is of critical value for\\
    complex statistical models\\[3ex]
    Open source principles (making a thorough description\\
    of methods publicly available) have been a foundation\\
    of scientific research for
    centuries\\[2ex]
  \end{itemize}
\end{frame}

% ______________________________________________________________________________

\begin{frame}{Repeatable and reviewable research}
  \begin{itemize}
    \item[] An Excel file containing a complex analysis may not allow\\
    other scientists to apply the method to another dataset, or\\
    to understand the different steps in the analysis\\[6ex]
    \item[] R makes a clear separation between data and calculations, and\\
    the computations follow a logical order: first this, then that, etc.\\[6ex]
    \item[] We try to write code that is easy for others to understand\\[4ex]
  \end{itemize}
\end{frame}

% ______________________________________________________________________________

\begin{frame}{Scientific method}
  \begin{itemize}
    \item[] Open source statistical software has become a cornerstone of\\
    scientific inference, and is a modern element of the scientific method:\\
    medical research, astronomy, everywhere.\\[6ex]
    \item[] The software development process is a collaborative effort of
    scientists\\
    worldwide, and relies on users contributing code, documentation,\\
    bug reports, etc.\\[6ex]
    \item[] You can contribute to open source software\\[2ex]
  \end{itemize}
\end{frame}

% ______________________________________________________________________________

\begin{frame}{First steps in R}
  \textbf{Install \blue R}\\[1ex]
  \begin{itemize}
    \item[] \url{https://www.r-project.org}\\[8ex]
  \end{itemize}
  \textbf{Install \green RStudio} (or some other editor)\\[1ex]
  \begin{itemize}
    \item[] \url{https://posit.co}\\[3ex]
  \end{itemize}
\end{frame}

% ______________________________________________________________________________

\begin{frame}[fragile]{Calculator with functions}
\begin{semiverbatim}
2 + 2


sqrt(10)


log(10)


\textgray{# try the up arrow}
\end{semiverbatim}
\end{frame}

% ______________________________________________________________________________

\begin{frame}[fragile]{Objects in workspace}
\begin{semiverbatim}
x <- 2

10 * x


ls()

rm(x)

rm(list=ls())
\end{semiverbatim}
\end{frame}

% ______________________________________________________________________________

\begin{frame}[fragile]{Data objects}
  \textblue{Vectors}
  \begin{itemize}
    \item[] \verb|rivers|
    \item[] \verb|month.abb|\\[4ex]
  \end{itemize}
  \textblue{Data frames}
  \begin{itemize}
    \item[] \verb|BOD|
    \item[] \verb|mtcars|\\[4ex]
  \end{itemize}
  \textblue{Select column}
  \begin{itemize}
    \item[] \verb|mtcars$hp|
  \end{itemize}
  \vspace{2ex}
\end{frame}

% ______________________________________________________________________________

\begin{frame}[fragile]{Plots}
\begin{semiverbatim}
x <- 1:10

y <- 3 * x


plot(x, y)


y <- 100 * x

\end{semiverbatim}
  The plot is not ``alive'', so the y coordinates are not updated\\[0.5ex]
  unless \verb|plot(x, y)| is called again\\[3ex]
\end{frame}

% ______________________________________________________________________________

\begin{frame}[fragile]{Help}
\begin{semiverbatim}
help(log)

?log



args(log)
\end{semiverbatim}
\end{frame}

% ______________________________________________________________________________

\begin{frame}{Help}
  If you get an error message:
  \begin{itemize}
    \item[] press the up arrow and try to rewrite
    \item[] the error message sometimes describes the problem\\[6ex]
  \end{itemize}
  If R doesn't respond to user input:
  \begin{itemize}
    \item[] press the \texttt{Esc} key
  \end{itemize}
\end{frame}

% ______________________________________________________________________________

\begin{frame}{Outline}
  \textbf{\darkblue Statistical Computing and R}\\
  \comment{features, history}\\[4ex]
  \textbf{\darkblue Open Science}\\
  \comment{software, scientific method, repeatability, reviewability}\\[4ex]
  \textbf{\darkblue First Steps in R}\\
  \comment{calculator, objects, plots, help}\\[4ex]
\end{frame}

\end{document}

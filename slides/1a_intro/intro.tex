\documentclass[aspectratio=169]{beamer}
\usepackage[utf8]{inputenc}
\usepackage[T1]{fontenc}
\usepackage{generic}
\begin{document}

\begin{frame}
  \title{\vspace{-2ex}\darkblue Course Overview}
  \author{\vspace{-16ex}\normalsize Arni Magnusson}
  \date{\vspace{-3ex}
    {\darkgreen\it Statistical Modeling in R}\\[0.5ex]
    Universidad de Concepción\\[0.5ex]
    19\h{0.2ex}--\h{0.1ex}23 January 2026}
  \titlepage
\end{frame}

% ______________________________________________________________________________

\begin{frame}{Overview}
  \begin{enumerate}
    \item \textbf{Introduction to R}\\
    \comment{data, plots, tests, linear models, projects, help, functions,
      packages}\\[2ex]
    \item \textbf{Extending the Linear Model}\\
    \comment{generalized linear models, additive models, mixed effects}\\[2ex]
    \item \textbf{Nonlinear Models}\\
    \comment{uncertainty, maximum likelihood, hessian, simulations}\\[2ex]
    \item \textbf{RTMB}\\
    \comment{automatic differentiation, writing models, running models}\\[2ex]
    \item \textbf{Software Development}\\
    \comment{interacting with other programs, writing packages, github}\\[3ex]
  \end{enumerate}
\end{frame}

% ______________________________________________________________________________

\begin{frame}{Teach yourself programming in 10 years}
  \textgray{\it Famous essay by Peter Norvig, a computer scientist}\\[3ex]
  \begin{itemize}
    \item[] It takes around 10 years to develop expertise in a given field:\\
    chess, painting, tennis, swimming, piano, programming, \dots\\[5ex]
    \item[] Use the programming languages that your friends use\\[5ex]
    \item[] Work with other programmers, contribute to open source
    projects\\[5ex]
  \end{itemize}
\end{frame}

% ______________________________________________________________________________

\begin{frame}{Learning goals and AI}
  \begin{itemize}
    \item[] It's important to think about learning goals\\[4ex]
    \item[] You will probably work with statistical models in your career\\[4ex]
    \item[] How can you benefit the most from this course?\\[4ex]
    \item[] AI can be useful, but it can also slow down your learning\\[4ex]
    \item[] You can do well in this course without using AI\\[2ex]
  \end{itemize}
\end{frame}

% ______________________________________________________________________________

\begin{frame}{Organizing your notes and code this week}
  \textgray{\it Some ideas \dots}\\[3ex]
  \begin{itemize}
    \item[] Take personal {\bf\blue course notes} (day by day), either in a word
    processor,\\
    text editor, or just on paper.\\[4ex]
    \item[] Organize {\bf\blue long-term notes} (by subject) in a word processor
    or text editor.\\
    You can gradually build these notes over months and years,
    and use them\\
    as a reference.\\[4ex]
    \item[] Organize and save your work as {\bf\blue R scripts}. Scripts are
    text files that\\
    contain code (and often comments) that can be pasted or sourced into
    R.\\[3ex]
  \end{itemize}
\end{frame}

% ______________________________________________________________________________

\begin{frame}{Format of the course}
  \textblue{Challenges}\\[0.5ex]
  \begin{itemize}
    \item A lot of statistics and programming concepts in a short time\\
    $\Rightarrow$ may feel too fast and confusing\\[1ex]
    \item Participants have different background in statistical computing
  \end{itemize}
  \vspace{2ex}
  \textblue{Approach}\\[0.5ex]
  \begin{itemize}
    \item Emphasis on exercises and discussion rather than lectures\\[1ex]
    \item Open-ended exercises $\Rightarrow$ participants can work on basic or\\
    advanced aspects, depending on their background\\[1ex]
    \item Bring your own exercises and projects to class\\[1ex]
    \item Work together in pairs as much as we can; we gain deeper\\
    insight when discussing with others
  \end{itemize}
\end{frame}

% ______________________________________________________________________________

\begin{frame}{Introduce ourselves}
  \begin{itemize}
    \item[] \textblue{Name} and \textblue{workplace}\\[6ex]
    \item[] \textblue{Project(s)} you are working on\\[6ex]
    \item[] Previous \textblue{background} in R and statistical computing in
    general\\[6ex]
    \item[] What you hope to \textblue{learn} this week\\[4ex]
  \end{itemize}
\end{frame}

% ______________________________________________________________________________

\begin{frame}{Course evaluation}
  \textblue{\bf R scripts that you create (75\%)}\\[1ex]
  \quad {\darkgray\it from Monday, Tuesday, and Wednesday}\\[8ex]
  \textblue{\bf Group project (25\%)}\\[1ex]
  \quad {\darkgray\it present on Friday}
\end{frame}

% ______________________________________________________________________________

\begin{frame}{Course evaluation}
  \textblue{\bf R scripts that you create (75\%)}\\[2.5ex]
  At the end of each day, email me all R code that you wrote that day. The R
  scripts should:\\[1ex]
  \begin{itemize}
    \item Reach me before class starts the next day.\\[2ex]
    \item Run on my computer without returning an error, but if you want you
    can also send\\
    a separate file with R code that doesn't quite
    work.\\[2ex]
    \item Not include unnecessary code.\\[2ex]
    \item Include some comments describing the purpose of each part of the
    code. Well organized code doesn't need a lot of comments. The comments
    should enable an experienced R programmer to read quickly through the
    script and understand it.
  \end{itemize}
\end{frame}

% ______________________________________________________________________________

\begin{frame}{Course evaluation}
  \textblue{\bf Group project (25\%)}\\[2.5ex]
  \begin{itemize}
    \item Form groups and discuss datasets on Wednesday.\\[2ex]
    \item Analyze a dataset and write R functions.\\[2ex]
    \item Present on Friday.
  \end{itemize}
\end{frame}

% ______________________________________________________________________________

\begin{frame}{Overview}
  \begin{enumerate}
    \item \textbf{Introduction to R}\\
    \comment{data, plots, tests, linear models, projects, help, functions,
      packages}\\[2ex]
    \item \textbf{Extending the Linear Model}\\
    \comment{generalized linear models, additive models, mixed effects}\\[2ex]
    \item \textbf{Nonlinear Models}\\
    \comment{uncertainty, maximum likelihood, hessian, simulations}\\[2ex]
    \item \textbf{RTMB}\\
    \comment{automatic differentiation, writing models, running models}\\[2ex]
    \item \textbf{Software Development}\\
    \comment{interacting with other programs, writing packages, github}\\[3ex]
  \end{enumerate}
\end{frame}

\end{document}
